\documentclass[a4paper,11pt]{article}

\usepackage[francais]{babel}
\usepackage[utf8]{inputenc}
\usepackage[T1]{fontenc}
\usepackage{lmodern}
\usepackage{amsmath}
\usepackage{graphicx}
\usepackage{amssymb}
\usepackage{geometry}
\usepackage[absolute]{textpos}
\usepackage{listings}
\usepackage{eurosym}
\usepackage[colorlinks=true,linkcolor=black]{hyperref}


\begin{document}
% TODO objet
% TODO références client


Bonjour,

J'ai récemment résilié mon abonnement ADSL à la Box de SFR. J'ai restitué mon matériel en suivant la procédure qui m'a été communiquée par SFR : j'ai emballé et protégé mon matériel (box et décodeur) dans un colis expédié avec l'étiquette fournie par SFR. Cet envoi a été effectué à la Poste le 12 avril 2014.

Cependant, le 19 mai 2014, j'ai reçu deux SMS qui affirment que le matériel n'a pas été retourné, et que je vais être facturé 60\euro{} + 210\euro{}. Je vous prie de bien vouloir trouver la preuve du dépôt du colis en pièce jointe, avec le cachet de la poste pour preuve.

\textbf{Je souhaite donc ne pas être débité de cette somme car j'ai restitué mes équipement en parfait état et en respectant les règles.}

Je vous fais par de mon fort mécontentement, car à chaque étape de ma résiliation, des erreurs ont été commises par SFR :
\begin{itemize}
 \item non prise en compte du changement de titulaire sur la ligne ADSL ;
 \item envoi des étiquettes de retour matériel à la mauvaise adresse ;
 \item facturation à tort du matériel.
\end{itemize}

Je vous prie bien vouloir me tenir au courant de l'évolution de la situation.

\bigskip

Bastien Bonnet

\end{document}
